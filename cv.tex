\documentclass[10pt]{article}

\usepackage{array}
\usepackage{xcolor}
\usepackage{bibentry}
\makeatletter\let\saved@bibitem\@bibitem\makeatother
\usepackage[margin=2.5cm]{geometry}
\usepackage{longtable}
\usepackage{needspace}
\usepackage{multicol}
\usepackage[T1]{fontenc}
\usepackage[super]{nth}
\usepackage{hyperref}
\usepackage{fancyhdr}
\usepackage{lastpage}
\makeatletter\let\@bibitem\saved@bibitem\makeatother

\newcommand{\name}{Lorenzo Stella}

% DATA %%%%%%%%%%%%%%%%%%%%%%%%%%%%%%%%%%%%%%%%%%%%%%%%%%%%%%%%%%%%%
\newcommand{\born}{December 12, 1985}
\newcommand{\bornin}{Florence, Italy}
\newcommand{\citizenship}{Italian, American}

% ADDRESS LUCCA %%%%%%%%%%%%%%%%%%%%%%%%%%%%%%%%%%%%%%%%%%%%%%%%%%%%%%
%\newcommand{\addressCO}{IMT Lucca}
%\newcommand{\addressSTREET}{Piazza San Francesco, 19}
%\newcommand{\addressCITY}{55100 Lucca, Italy}

% ADDRESS LEUVEN %%%%%%%%%%%%%%%%%%%%%%%%%%%%%%%%%%%%%%%%%%%%%%%%%%%%%
\newcommand{\addressCO}{ESAT, KU Leuven}
\newcommand{\addressSTREET}{Kasteelpark Arenberg 10}
\newcommand{\addressCITY}{3001 Leuven, Belgium}

% EMAIL, PHONE %%%%%%%%%%%%%%%%%%%%%%%%%%%%%%%%%%%%%%%%%%%%%%%%%%%%%%%
\newcommand{\emailONE}{{\tt lorenzo.stella@imtlucca.it}}
\newcommand{\emailTWO}{{\tt lstella@esat.kuleuven.be}}
\newcommand{\email}{{\tt lorenzostella@gmail.com}}
%\newcommand{\mobile}{+39 331 54 51 166}

% OTHER COMMANDS, ENVIRONMENTS %%%%%%%%%%%%%%%%%%%%%%%%%%%%%%%%%%%%%%%
\newcommand{\spacednewline}{\\[5pt]}
\newcommand{\subtitle}[1]{\hfill\textnormal{\footnotesize #1}}
\newcommand{\lang}[1]{{\sc #1}}
\newcommand{\myurl}[1]{{\tt #1}}
\newcommand{\mygithub}{\href{https://github.com/lostella}{\myurl{github.com/lostella}}}
\newcommand{\myscholar}{\href{http://scholar.google.com/citations?user=Y3ag8YsAAAAJ}{\myurl{scholar.google.com/citations?user=Y3ag8YsAAAAJ}}}
\newcommand{\mywebpage}{\href{https://lostella.github.io}{\myurl{lostella.github.io}}}
\newcommand{\myarxiv}{\href{http://arxiv.org/find/math/1/au:+Stella_L/0/1/0/all/0/1}{\myurl{arxiv.org/find/math/1/au:+Stella_L/0/1/0/all/0/1}}}

\definecolor{lightgray}{gray}{0.6}
\newcolumntype{L}{>{\raggedleft}p{0.15\textwidth}}
\newcolumntype{R}{p{0.85\textwidth}}
\newcommand\VRule{\color{lightgray}\vrule width 0.5pt}

\newenvironment{cvsection}[1]
    {\centering\begin{longtable}{L!{\VRule}R} \multicolumn{1}{r}{} & \multicolumn{1}{l}{\large \bf #1}\\[5pt]}
    {\end{longtable}}
    
\newcommand{\cvspace}{\vspace{15pt}}

% FANCYHDR %%%%%%%%%%%%%%%%%%%%%%%%%%%%%%%%%%%%%%%%%%%%%%%%%%%%%%%%%%%
\pagestyle{fancy}
\fancyhf{}
\cfoot{\color{lightgray}{Page \thepage\ of \pageref{LastPage}}}
\renewcommand{\headrulewidth}{0pt}

 
\begin{document}

%{\Huge\name}

%\vspace{20pt}

\begin{minipage}[t]{0.28\linewidth}
	{\huge\name}
%	\small
%	\begin{tabular}{rl}
%		{\color{lightgray}Born}	 	& December 12, 1985 \\
%		{\color{lightgray}In}		& Florence (Italy) \\
%		{\color{lightgray}Nationality}	& American, Italian \\
%	\end{tabular}
\end{minipage}\hfill
\begin{minipage}[t]{0.3\linewidth}
	\small
	\begin{tabular}{rl}
		{\color{lightgray}Address} 	& \addressCO \\
   		 	                       	& \addressSTREET \\
        		                   	& \addressCITY \\
	\end{tabular}
\end{minipage}\hfill
\begin{minipage}[t]{0.32\linewidth}
	\small
	\begin{tabular}{rl}
		{\color{lightgray}Email} 	& \emailONE \\
									& \emailTWO \\
		{\color{lightgray}Web}		& \mywebpage
	\end{tabular}
\end{minipage}

\vspace{10pt}

\begin{cvsection}{Professional Experience}
Feb 2013 -- now & PhD student at IMT School for Advanced Studies, Lucca (Italy)\hfill\href{http://www.imtlucca.it}{\myurl{www.imtlucca.it}}\\
	& and KU Leuven, Leuven (Belgium).\hfill\href{http://www.esat.kuleuven.be/stadius}{\myurl{www.esat.kuleuven.be/stadius}}\spacednewline
    & Nonsmooth optimization algorithms, applications to optimal control, distributed optimization, large-scale machine learning, image processing.
    Teaching assistant, exercises and laboratory sessions for the ``Optimization'' class, taught by Panos Patrinos, at KU Leuven.\spacednewline
2011 -- 2012 & Research Analyst at COSBI, Trento (Italy).\hfill\href{http://www.cosbi.eu}{\myurl{www.cosbi.eu}}\spacednewline
    & Analysis and simulation of stochastic models in systems biology (PK/PD, metabolic networks). Inference and analysis of gene regulatory networks. Development of tools
	for stochastic simulation and network analysis in \lang{C\#}, \lang{Python}\ and \lang{Matlab}\ languages.
\end{cvsection}

\begin{cvsection}{Education}
2008 -- 2011 & M.S. in Computer Science, University of Florence. Final grade: 110/110 cum laude.\spacednewline
	& Thesis supervised by Prof. Luigi Brugnano, \emph{Efficient methods for the numerical solution of Hamiltonian problems}.
	Analysis of the effectiveness of numerical methods for ODEs with respect to the conservation of energy in the case of Hamiltonian systems. Efficient implementation of such techniques using a framework developed in \lang{C}.\spacednewline
2004 -- 2008 & B.S. in Computer Science, University of Florence. Final grade: 110/110.\spacednewline
	& Thesis supervised by Prof. Luigi Brugnano, \emph{Numerical methods in Linear Algebra with applications to Google's Pagerank}. Study of the \emph{random surfer} model and possible approaches to the computation of the stationary point of the associated Markov chain. Experimental results obtained with \lang{Matlab} simulations.
\end{cvsection}

\bibliographystyle{siam}
\nobibliography{publications.bib}

\begin{cvsection}{Publications \subtitle{Google Scholar: \myscholar}}
2016	& \bibentry{themelis16forward}\spacednewline
		& \bibentry{stella16forward}\spacednewline
		& \bibentry{latafat16new}\spacednewline
2014    & \bibentry{patrinos14douglas}\spacednewline
        & \bibentry{patrinos14forward}
\end{cvsection}

\begin{cvsection}{Talks and seminars}
Nov. 2015 & ``Proximal quasi-Newton methods for nonsmooth composite optimization problems,'' at the KU Leuven Optimization in Engineering Center (OPTEC), Spa, Belgium. \spacednewline
Jul. 2015 & ``Accelerated L-BFGS for large scale nonsmooth convex optimization,'' at the \nth{22} International Symposium on Mathematical
    Programming (ISMP 2015), Pittsburgh, PA, USA. \spacednewline
Dec. 2014 & ``Douglas-Rachford splitting: complexity estimates and accelerated variants,'' at the \nth{53} IEEE Conference on Decision
    and Control (CDC 2014), Los Angeles, CA, USA.
\end{cvsection}

\begin{cvsection}{Software projects \subtitle{GitHub: \mygithub}}
ForBES		& \lang{Matlab}\ solver for nonsmooth optimization problems, provided with a library of mathematical functions used to model problems arising in numerous application fields such as control, machine learning, image and signal processing.\spacednewline
			& Web page: \href{https://kul-forbes.github.io/ForBES}{\myurl{kul-forbes.github.io/ForBES}} \spacednewline
libForBES	& \lang{C++} framework for modeling and solving large-scale nonsmooth optimization problems, will allow to interface many high-level languages (including \lang{R}, \lang{Python}, \lang{Julia}) to a unique solver capable of addressing nonsmooth optimization problems from several application fields.\spacednewline
			& Web page: \href{https://kul-forbes.github.io/libForBES}{\myurl{kul-forbes.github.io/libForBES}} \spacednewline
libLBFGS	& \lang{C} library providing the structures and routines for implementing the limited-memory BFGS algorithm (L-BFGS) for large-scale smooth unconstrained optimization. Contains a \lang{Mex} interface to \lang{Matlab}.\spacednewline
			& Web page: \href{https://github.com/lostella/libLBFGS}{\myurl{github.com/lostella/libLBFGS}}
\end{cvsection}

\begin{cvsection}{Programming skills}
Proficient & \lang{C}, \lang{MATLAB}, \lang{Python}, \lang{Julia}, \lang{Java}, \lang{C++} \spacednewline
Familiar & \lang{C\#}, \lang{Haskell}
\end{cvsection}

\begin{cvsection}{Languages}
English & Native\spacednewline
Italian & Native
\end{cvsection}

\centering{\color{lightgray}\footnotesize{Updated \today}}

\end{document}
