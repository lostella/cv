\documentclass[10pt]{article}

\usepackage{array}
\usepackage{xcolor}
\usepackage{bibentry}
\makeatletter\let\saved@bibitem\@bibitem\makeatother
\usepackage[margin=2.5cm]{geometry}
\usepackage{longtable}
\usepackage{needspace}
\usepackage{multicol}
\usepackage[T1]{fontenc}
\usepackage[super]{nth}
\usepackage{hyperref}
\usepackage{fancyhdr}
\usepackage{lastpage}
\makeatletter\let\@bibitem\saved@bibitem\makeatother

\newcommand{\name}{Lorenzo Stella}

% DATA %%%%%%%%%%%%%%%%%%%%%%%%%%%%%%%%%%%%%%%%%%%%%%%%%%%%%%%%%%%%%
\newcommand{\born}{December 12, 1985}
\newcommand{\bornin}{Florence, Italy}
\newcommand{\citizenship}{Italian, American}

% ADDRESS LUCCA %%%%%%%%%%%%%%%%%%%%%%%%%%%%%%%%%%%%%%%%%%%%%%%%%%%%%%
%\newcommand{\addressCO}{IMT Lucca}
%\newcommand{\addressSTREET}{Piazza San Francesco, 19}
%\newcommand{\addressCITY}{55100 Lucca, Italy}

% ADDRESS LEUVEN %%%%%%%%%%%%%%%%%%%%%%%%%%%%%%%%%%%%%%%%%%%%%%%%%%%%%
\newcommand{\addressCO}{ESAT, KU Leuven}
\newcommand{\addressSTREET}{Kasteelpark Arenberg 10}
\newcommand{\addressCITY}{3001 Leuven, Belgium}

% EMAIL, PHONE %%%%%%%%%%%%%%%%%%%%%%%%%%%%%%%%%%%%%%%%%%%%%%%%%%%%%%%
\newcommand{\emailONE}{{\tt lorenzo.stella@imtlucca.it}}
\newcommand{\emailTWO}{{\tt lstella@esat.kuleuven.be}}
\newcommand{\email}{{\tt lorenzostella@gmail.com}}
%\newcommand{\mobile}{+39 331 54 51 166}

% OTHER COMMANDS, ENVIRONMENTS %%%%%%%%%%%%%%%%%%%%%%%%%%%%%%%%%%%%%%%
\newcommand{\spacednewline}{\\[5pt]}
\newcommand{\subtitle}[1]{\hfill\textnormal{\footnotesize #1}}
\newcommand{\lang}[1]{{\sc #1}}
\newcommand{\myurl}[1]{{\tt #1}}
\newcommand{\mygithub}{\href{https://github.com/lostella}{\myurl{github.com/lostella}}}
\newcommand{\myscholar}{\href{http://scholar.google.com/citations?user=Y3ag8YsAAAAJ}{\myurl{scholar.google.com/citations?user=Y3ag8YsAAAAJ}}}
\newcommand{\mywebpage}{\href{https://lostella.github.io}{\myurl{lostella.github.io}}}
\newcommand{\myarxiv}{\href{http://arxiv.org/find/math/1/au:+Stella_L/0/1/0/all/0/1}{\myurl{arxiv.org/find/math/1/au:+Stella_L/0/1/0/all/0/1}}}

\definecolor{lightgray}{gray}{0.6}
\newcolumntype{L}{>{\raggedleft}p{0.15\textwidth}}
\newcolumntype{R}{p{0.85\textwidth}}
\newcommand\VRule{\color{lightgray}\vrule width 0.5pt}

\newenvironment{cvsection}[1]
    {\centering\begin{longtable}{L!{\VRule}R} \multicolumn{1}{r}{} & \multicolumn{1}{l}{\large \bf #1}\\[5pt]}
    {\end{longtable}}
    
\newcommand{\cvspace}{\vspace{15pt}}

% FANCYHDR %%%%%%%%%%%%%%%%%%%%%%%%%%%%%%%%%%%%%%%%%%%%%%%%%%%%%%%%%%%
\pagestyle{fancy}
\fancyhf{}
\cfoot{\color{lightgray}{Page \thepage\ of \pageref{LastPage}}}
\renewcommand{\headrulewidth}{0pt}

 
\begin{document}

\begin{minipage}[ht]{0.60\textwidth}
{\Huge \name}
\end{minipage}
\begin{minipage}[ht]{0.40\textwidth}
\begin{tabular}{rl}
{\color{lightgray}Address} & \addressCO \\
                           & \addressSTREET \\
                           & \addressCITY \\
{\color{lightgray}Email} & \email \\
{\color{lightgray}Mobile} & \mobile
\end{tabular}
\end{minipage}
\vspace{50pt}

\begin{cvsection}{Professional Experience}
2013 -- now & PhD student at IMT Institute for Advanced Studies Lucca (www.imtlucca.it), Lucca (Italy).\\[5pt]
    & Convex analysis and numerical methods in convex optimization, with particular
    focus on splitting methods for the solution of nonsmooth convex composite problems.
    Applications to optimal control problems, distributed optimization and large-scale
    problems arising in machine learning, image processing.\\[5pt]
2011 -- 2012 & Research Analyst at COSBI (www.cosbi.eu), Trento (Italy).\\[5pt]
    & Analysis and simulation of stochastic models in systems biology (PK/PD, metabolic networks). Inference and analysis of gene regulatory networks. Development of tools
	for stochastic simulation and network analysis in \csharp, \python\ and \matlab\ languages.\\[5pt]
2011 & Software Developer at WorkHard (www.workhard.ph), Florence (Italy).\\[5pt]
    & Multimedia software development using \asthree\ language for the \flash/\air\ platform. Interaction with RFID devices.
	Experiments with \csharp\ and the Microsoft Kinect SDK.
\end{cvsection}

\begin{cvsection}{Education}
2008 -- 2011 & MSc in Computer Science, University of Florence, 110/110 cum laude.\\[5pt]
    & Thesis, supervised by Prof. Luigi Brugnano:\\[5pt]
	& \multicolumn{1}{c}{\textsc{Efficient methods for the numerical solution of Hamiltonian problems}}\\[5pt]
	& Analysis of the effectiveness of numerical methods for differential equations with respect to the preservation
	of qualitative properties of the simulated system, with particular attention to energy conservation in the case
	of Hamiltonian systems and to the efficient implementation of such techniques, using a framework developed \emph{ad hoc}
	with the \clang\ language.\\[5pt]
%    & Courses attended:
%    \begin{multicols}{2}
%    \begin{itemize}
%        \item Algorithm Design
%        \item Algorithms for Networks
%        \item Analysis of Algorithms
%   		\item Approximation Methods I
%   		\item Approximation Methods II
%        \item Combinatorial Modeling
%        \item Data Structures for Databases
%   		\item Distributed Databases
%   		\item Fuzzy Logic
%		\item Information Theory
%		\item Numerical Analysis
%		\item Numerical Methods for CAGD
%		\item Numerical Optimization
%		\item Operational Research
%	\end{itemize}
%    \end{multicols}\\[5pt]
2004 -- 2008 & BSc in Computer Science, University of Florence, 110/110.\\[5pt]
    & Thesis, supervised by Prof. Luigi Brugnano:\\[5pt]
    & \multicolumn{1}{c}{\textsc{Numerical methods in Linear Algebra with applications}}\\[0pt]
    & \multicolumn{1}{c}{\textsc{to Google's Pagerank}}\\[5pt]
	& Study of the \emph{random surfer} model and possible approaches to the calculation of the stationary point of the Markov
	chain associated with it, with the aim of combining modeling and mathematical aspects of the problem with those of its efficient
	resolution on a computer. The approaches and algorithms presented were compared on the basis of experimental results obtained
	with \matlab\ implementations.
\end{cvsection}

\bibliographystyle{plain}
\nobibliography{publications.bib}

\begin{cvsection}{Publications}
2014    & \bibentry{patrinos14douglas}\\ \\
        & \bibentry{patrinos14forward}\\ \\
2013    & \bibentry{scotti13modeling}
\end{cvsection}

\begin{cvsection}{Talks and seminars}
Jul. 2015 & \emph{Accelerated L-BFGS for large scale nonsmooth convex optimization}, at \nth{22} International Symposium on Mathematical
    Programming (ISMP 2015), Pittsburgh, PA, USA. \\[5pt]
Dec. 2014 & \emph{Douglas-Rachford splitting: complexity estimates and accelerated variants}, at \nth{53} IEEE Conference on Decision
    and Control (CDC 2014), Los Angeles, CA, USA.
\end{cvsection}

\begin{cvsection}{Scientific and technical skills}
Mathematics & Algebra, Linear Algebra, Calculus, Numerical Analysis, Probability, Geometry, Dynamical Systems, Mathematical Logic.\\[5pt]
Computer Science & Algorithms, Data Structures, Theoretical Computer Science and Computational Complexity, Databases.\\[5pt]
Programming & Excellent knowledge of \clang, \python, \matlab, \java. Good skills in \cplusplus,
\csharp. Familiar with \scheme, \haskell, \perl, \fortran.\\[5pt]
Operating systems & \gnulinux. Good knowledge of \macosx, \mswin.\\[5pt]
Other tools & \LaTeX, \mysql\ database management system.
\end{cvsection}

\begin{cvsection}{Languages}
Italian & Native\\[5pt]
English & Fluent
\end{cvsection}

\vspace{2em}
\centering{\color{lightgray}\footnotesize{Updated \today}}

\end{document}
