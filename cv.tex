\documentclass[10pt]{article}

\usepackage{array, xcolor, bibentry}
\usepackage[margin=3cm]{geometry}
\usepackage{longtable}
\usepackage{needspace}
\usepackage{multicol}
\usepackage{libertine}
\usepackage[T1]{fontenc}

\newcommand{\name}{Lorenzo Stella}
\newcommand{\addressCO}{C/O IMT Lucca}
\newcommand{\addressSTREET}{Piazza San Francesco, 19}
\newcommand{\addressCITY}{55100 Lucca, Italy}
%\newcommand{\email}{lorenzostella@gmail.com}
\newcommand{\email}{lorenzo.stella@imtlucca.it}
\newcommand{\mobile}{+39 331 54 51 166}

\newcommand{\matlab}{{MATLAB}}
\newcommand{\clang}{{C}}
\newcommand{\cplusplus}{{C++}}
\newcommand{\csharp}{{C\#}}
\newcommand{\python}{{Python}}
\newcommand{\perl}{{Perl}}
\newcommand{\haskell}{{Haskell}}
\newcommand{\fortran}{{Fortran}}
\newcommand{\scheme}{{Scheme}}
\newcommand{\java}{{Java}}
\newcommand{\gnulinux}{{GNU/Linux}}
\newcommand{\macosx}{{Mac OS X}}
\newcommand{\mswin}{{MS Windows}}
\newcommand{\mysql}{{MySQL}}
\newcommand{\flash}{{Flash}}
\newcommand{\air}{{AIR}}
\newcommand{\asthree}{{ActionScript3}}

\definecolor{lightgray}{gray}{0.8}
\newcolumntype{L}{>{\raggedleft}p{0.2\textwidth}}
\newcolumntype{R}{p{0.8\textwidth}}
\newcommand\VRule{\color{lightgray}\vrule width 0.5pt}

\newenvironment{cvsection}[1]
    {\needspace{5\baselineskip}\begin{longtable}{L!{\VRule}R} \multicolumn{1}{r}{} & \multicolumn{1}{l}{\Large #1}\\[10pt]}
    {\end{longtable}}

\pagestyle{empty}


 
\begin{document}

%{\Huge\name}

%\vspace{20pt}

\begin{minipage}[t]{0.28\linewidth}
	{\huge\name}
%	\small
%	\begin{tabular}{rl}
%		{\color{lightgray}Born}	 	& December 12, 1985 \\
%		{\color{lightgray}In}		& Florence (Italy) \\
%		{\color{lightgray}Nationality}	& American, Italian \\
%	\end{tabular}
\end{minipage}\hfill
\begin{minipage}[t]{0.3\linewidth}
	\small
	\begin{tabular}{rl}
		{\color{lightgray}Address} 	& \addressCO \\
   		 	                       	& \addressSTREET \\
        		                   	& \addressCITY \\
	\end{tabular}
\end{minipage}\hfill
\begin{minipage}[t]{0.32\linewidth}
	\small
	\begin{tabular}{rl}
		{\color{lightgray}Email} 	& \emailONE \\
									& \emailTWO \\
		{\color{lightgray}Web}		& \mywebpage
	\end{tabular}
\end{minipage}

\vspace{20pt}

\begin{cvsection}{Professional Experience}
Sept 2015 -- now & Visiting PhD student at KU Leuven, Leuven (Belgium),\\
	& Departement Elektrotechniek (ESAT),\\
	& Stadius division.\hfill\href{http://www.esat.kuleuven.be/stadius}{\myurl{www.esat.kuleuven.be/stadius}}\spacednewline
	& Teaching assistant, exercises and laboratory sessions for the Optimization class (\mbox{H03E3a}, taught by Panos Patrinos).\spacednewline
Feb 2013 -- now & PhD student at IMT School for Advanced Studies, Lucca (Italy).\hfill\href{http://www.imtlucca.it}{\myurl{www.imtlucca.it}}\spacednewline
    & Convex optimization, operator splitting methods. Derivation, analysis and implementation of line-search methods based on the concept of \emph{splitting envelope} function, to tackle nonsmooth (possibly constrained) optimization problems
    (both convex and nonconvex) with classical smooth unconstrained techniques.
    Applications to optimal control problems, distributed optimization and large-scale
    problems arising in machine learning, image processing.\spacednewline
2011 -- 2012 & Research Analyst at COSBI, Trento (Italy).\hfill\href{http://www.cosbi.eu}{\myurl{www.cosbi.eu}}\spacednewline
    & Analysis and simulation of stochastic models in systems biology (PK/PD, metabolic networks). Inference and analysis of gene regulatory networks. Development of tools
	for stochastic simulation and network analysis in \lang{C\#}, \lang{Python}\ and \lang{Matlab}\ languages.
\end{cvsection}

\begin{cvsection}{Education}
2008 -- 2011 & M.S. in Computer Science, University of Florence, 110/110 cum laude.\spacednewline
%    & Thesis, supervised by Prof. Luigi Brugnano:\spacednewline
%	& \multicolumn{1}{c}{\textsc{Efficient methods for the numerical solution }}\\[0pt]
%    & \multicolumn{1}{c}{\textsc{of Hamiltonian problems}}\spacednewline
	& Thesis supervised by Prof. Luigi Brugnano, \emph{Efficient methods for the numerical solution of Hamiltonian problems}.
	Analysis of the effectiveness of numerical methods for ODEs with respect to the conservation of energy in the case of Hamiltonian systems. Efficient implementation of such techniques, using a framework developed \emph{ad hoc} in \lang{C}.\spacednewline
%    & Courses attended:
%    \begin{multicols}{2}
%    \begin{itemize}
%        \item Algorithm Design
%        \item Algorithms for Networks
%        \item Analysis of Algorithms
%   		\item Approximation Methods I
%   		\item Approximation Methods II
%        \item Combinatorial Modeling
%        \item Data Structures for Databases
%   		\item Distributed Databases
%   		\item Fuzzy Logic
%		\item Information Theory
%		\item Numerical Analysis
%		\item Numerical Methods for CAGD
%		\item Numerical Optimization
%		\item Operational Research
%	\end{itemize}
%    \end{multicols}\spacednewline
2004 -- 2008 & B.S. in Computer Science, University of Florence, 110/110.\spacednewline
%    & Thesis, supervised by Prof. Luigi Brugnano:\spacednewline
%    & \multicolumn{1}{c}{\textsc{Numerical methods in Linear Algebra with applications}}\\[0pt]
%    & \multicolumn{1}{c}{\textsc{to Google's Pagerank}}\spacednewline
	& Thesis supervised by Prof. Luigi Brugnano, \emph{Numerical methods in Linear Algebra with applications to Google's Pagerank}. Study of the \emph{random surfer} model and possible approaches to the calculation of the stationary point of the Markov chain associated with it. The approaches and algorithms presented were compared on the basis of experimental results obtained with \lang{Matlab}\ implementations.
\end{cvsection}

\bibliographystyle{siam}
\nobibliography{publications.bib}

\begin{cvsection}{Publications \subtitle{Google Scholar: \myscholar}}
2016	& \bibentry{themelis16forward}\spacednewline
		& \bibentry{stella16forward}\spacednewline
		& \bibentry{latafat16new}\spacednewline
2014    & \bibentry{patrinos14douglas}\spacednewline
        & \bibentry{patrinos14forward}%\spacednewline
%2013    & \bibentry{scotti13modeling}
\end{cvsection}

\begin{cvsection}{Talks and seminars}
Nov. 2015 & ``Proximal quasi-Newton methods for nonsmooth composite optimization problems,'' at the KU Leuven Optimization in Engineering Center (OPTEC), Spa, Belgium. \spacednewline
Jul. 2015 & ``Accelerated L-BFGS for large scale nonsmooth convex optimization,'' at the \nth{22} International Symposium on Mathematical
    Programming (ISMP 2015), Pittsburgh, PA, USA. \spacednewline
Dec. 2014 & ``Douglas-Rachford splitting: complexity estimates and accelerated variants,'' at the \nth{53} IEEE Conference on Decision
    and Control (CDC 2014), Los Angeles, CA, USA.
\end{cvsection}

\begin{cvsection}{Software projects \subtitle{GitHub: \mygithub}}
ForBES		& Generic and efficient \lang{Matlab}\ solver for nonsmooth optimization problems. The solver is provided with a library of mathematical functions used to model problems arising in numerous application fields such as control, machine learning, image and signal processing. Each function embeds the operators which are relevant for optimization purposes (such as gradient and proximal mappings).\spacednewline
			& Web page: \href{http://kul-forbes.github.io/ForBES}{\myurl{kul-forbes.github.io/ForBES}} \spacednewline
libForBES	& Framework in \lang{C++}\ for modeling and solving large-scale nonsmooth optimization problems. Started as a low-level implementation of ForBES, to overcome the drawbacks of \lang{Matlab}, it will allow to interface many high-level languages (including \lang{R}, \lang{Python}, \lang{Julia}) to a unique solver capable of addressing nonsmooth optimization problems from several application fields.\spacednewline
			& Web page: \href{http://kul-forbes.github.io/libForBES}{\myurl{kul-forbes.github.io/libForBES}} \spacednewline
libLBFGS	& Library written in \lang{C} containing the structures and routines necessary for computing search directions in the limited-memory BFGS algorithm (L-BFGS), for large-scale smooth unconstrained optimization. Contains a \lang{Mex} interface to \lang{Matlab}.\spacednewline
			& Web page: \href{https://github.com/lostella/libLBFGS}{\myurl{github.com/lostella/libLBFGS}}
\end{cvsection}

\begin{cvsection}{Summary of technical skills}
Programming & \lang{C}, \lang{Python}, \lang{MATLAB} (expert).\spacednewline
			& \lang{C++}, \lang{Java}, \lang{Julia} (proficient). \spacednewline
			& \lang{C\#}, \lang{Haskell}, \lang{Scheme}, \lang{R} (prior experience).
\end{cvsection}

\begin{cvsection}{Languages}
English & Native\spacednewline
Italian & Native
\end{cvsection}

%\vspace{1em}
\centering{\color{lightgray}\footnotesize{Updated \today}}

\end{document}
